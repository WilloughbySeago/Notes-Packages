\documentclass[fleqn, a4paper, openany]{memoir}
\usepackage{NotesStyle}
\usepackage{NotesBoxes}
\usepackage{NotesMaths}

\usepackage{blindtext}
\usepackage{csquotes}

\title{Notes Packages}
\author{Willoughby Seago}
\date{March 18, 2021}
\subtitle{A Collection of Packages}
\subsubtitle{Some Documentation}

\newcommand{\package}[1]{{\textsf{\footnotesize #1}}}  % Typeset package names in this font

\usepackage{fancyvrb}  % Used for Verbatim environment
% Set parameters for fancyvrb such that theres no extra space above and below the Verbatim environment
\fvset{listparameters=\setlength{\topsep}{0pt}\setlength{\partopsep}{0pt}}

\usepackage[language=british, sorting=none]{biblatex}
\addbibresource{notes-packages-bib.bib}

\definecolor{highlight}{RGB}{100,0,100}

\begin{document}
    \frontmatter
    \titlepage
    \maketitle
    \begin{abstract}
        While at university I've been learning \LaTeX.
        At the end of first year I wrote a package, called \package{NotesPackage} for use in second year, it was essentially just a preamble that I used in all documents.
        At the end of second year that package was a mess and I wrote an updated version, called \package{NotesPackage2}.
        Now almost at the end of third year it is again a mess and I am writing new packages.
        The idea this time is to separate different aspects into different packages which can be loaded together or individually.
        The three packages so far are \package{NotesStyle}, which controls how the document looks (this document is written with it), \package{NotesBoxes} which has a bunch of different box styles for various reasons, and \package{NotesMaths}, which is simply a lot of commands for typesetting maths.
        Since these packages are for my personal use they are subject to change at any point and definitely aren't stable.
        Everything in these packages has been done (probably better) in other packages so use those instead.
    \end{abstract}
    \tableofcontents
    \mainmatter
    \part{The Packages}
    
    \chapter{NotesStyle}
    The \package{NotesStyle} package controls the layout of the page.
    It works based on the \package{memoir} \cite{memoir} document class.
    If a different document class is used it will cause an error.
    This document has been prepared using the the \package{NotesStyle} package.
    
    \package{NotesStyle} loads the following packagesd: \package{fontenc}, with the \verb!T1! option, \package{xcolor} \cite{xcolor}, \package{graphicx} \cite{graphicx}, \package{tikz} \cite{tikz}, \package{pgf-pie} \cite{pgf-pie}, \package{hyperref} \cite{hyperref}, and \package{cleveref} \cite{cleveref}.
    As well as this the \package{tikz} library \package{external} is loaded.
    
    \section{Document Setup}
    This package provides a few commands for starting a document.
    These include making a title page, writing and abstract, and an inner-title page.
    Most of the work of this package is behind the scenes setting the page layout, fonts, table of contents style, glossary style, etc.
    
    \subsection{Title Page}
    The command \verb!\titlepage! will create a title page. This command should be the second thing in the document, the first being the command \verb!\frontmatter!.
    The information contained on the title page is as follows
    \begin{itemize}
        \item Author's name, this can be set with \verb!\author{<name>}!, the same as the authors name for \verb!\maketitle!.
        \item Title, this can be set with \verb!\subtitle{<the title>}!, the same as the title for \verb!\maketitle!.
        \item Subtitle, this can be set with \verb!\subtitle{<the subtitle>}!. The default is \verb!Theoretical Physics!
        \item Subsubtitle, this can be set with \verb!\subsubtitle{<the subsubtitle>}!. The default is \verb!Course Notes!.
        \item Date, this can be set with \verb!\date{<the date>}!, the same as \verb!\maketitle!.
    \end{itemize}
    In the preamble of this document the code for the title page was
    \begin{Verbatim}[gobble=2]
        \title{Notes Packages}
        \author{Willoughby Seago}
        \date{March 18, 2021}
        \subtitle{A Collection of Packages}
        \subsubtitle{Some Documentation}
    \end{Verbatim}
    and the results of \verb!\titlepage! can be seen at the start of this document.

    \subsection{Inner Title Page}
    This is only to be used 
    The command \verb!\innertitlepage{<empty or path to image>}! will create an inner title page. 
    The inner title page starts with \verb!\maketitle!.
    It then includes a predefined abstract with the following text:
    \begin{Verbatim}[gobble=2]
        These are my notes from the course {\MakeLowercase\thetitle{}}.
        I took this course as a part of the theoretical physics degree
        at the University of Edinburgh.
        
        These notes were last updated at \printtime{} on \today{}.
        For notes on other topics see
        \url{https://github.com/WilloughbySeago/Uni-Notes}.
    \end{Verbatim}
    where \verb!\thetitle! fills in the value set with \verb!\title{<the title>}!.
    
    Below the abstract the image given in the initial command is placed. Choose something relevant to the document. If no image is provided then a placeholder is used.
    
    \subsection{Glossaries and Indexes}
    This package starts an index, a glossary, and a glossary of acronyms.
    Items can be added to the index in the usual way using \verb!\index{}!. Similarly we can add to the glossary using \verb!\glossary{}{}! and \verb!\glossary[acronym]{}{}!.
    
    \subsection{Colours}
    This package defines the colour \verb!highlight! which is the colour used for boxes, section numbers, the chapter pie charts, etc.
    The default is set to the RGB colour \verb!{243, 102, 25}! which is \definecolor{highlightdefault}{RGB}{243, 102, 25}\tikz{\fill[highlightdefault] (0, 0) rectangle (0.25, 0.25);}. In this document \verb!highlight! was redefined to \verb!{100, 0, 100}! which is \tikz{\fill[highlight] (0, 0) rectangle (0.25, 0.25);}.
    It is expected that this colour will be chosen individually for each document.
    
    \chapter{NotesMaths}
    The simplest of the packages is \package{NotesMaths}.
    It defines lots of commands that I use when writing maths.
    In this section these commands will be listed and examples given.
    Unless specified otherwise all commands in this chapter are math mode commands.

    \package{NotesMaths} loads the following packages: \package{amsfonts} \cite{amsfonts}, \package{amsmath} \cite{amsmath}, \package{bm} \cite{bm}, and \package{xparse} \cite{xparse}.
    
    
    \section{Sets}
    \subsection{Number Sets}
    In maths we often need to typeset various number sets.
    Most commonly the real numbers (\(\reals\)), integers (\(\integers\)), complex numbers (\(\complex\)), etc.
    The predefined sets (and a generic field) are
    \begin{align}
        &\text{Natural numbers:}~&\naturals,\qquad &\text{Integers:}~&\integers,\\
        &\text{Rational numbers:}~&\rationals,\qquad &\text{Real numbers:}~&\reals,\\
        &\text{Complex numbers:}~&\complex,\qquad &\text{Quaternions:}~&\quaternions,\\
        &\text{Octonions:}~&\octonions,\qquad &\text{Field:}~&\field.
    \end{align}
    Which can be accessed by
    \begin{Verbatim}[gobble=2]
        \naturals             \integers
        \rationals            \reals
        \complex              \quaternions
        \octonions            \field
    \end{Verbatim}

    The default style for these sets, as we have already seen, is black board bold\footnote{the precise font is actually from \package{newtxmath} which has three blackboard bold fonts \cite{newtx}. This is the third which can be accessed as \texttt{\textbackslash{}vvmathbb}.}.
    This can be changed, suppose you want upright bold number sets then by doing either of
    \begin{Verbatim}[gobble=2]
        \renewcommand{\numset}[1]{\mathbf{#1}}
        \let\numset\mathbf
    \end{Verbatim}
    In which case the result will be
    \let\oldnumset\numset
    \let\numset\mathbf
    \begin{align}
        &\text{Natural numbers:}~&\naturals,\qquad &\text{Integers:}~&\integers,\\
        &\text{Rational numbers:}~&\rationals,\qquad &\text{Real numbers:}~&\reals,\\
        &\text{Complex numbers:}~&\complex,\qquad &\text{Quaternions:}~&\quaternions,\\
        &\text{Octonions:}~&\octonions,\qquad &\text{Field:}~&\field.
    \end{align}
    \let\numset\oldnumset
    
    \subsection{Matrix Sets}
    As well as sets of numbers it is common to deal with sets of matrices.
    Typically these have understood names such as `the general linear group', and predefined symbols to go along with them, in this case \(\generalLinear\).
    The most general set of matrices is the set of all \(m\times n\) matrices over a given field, \(\field\)\footnote{I suppose we could be even more general and consider matrices over a ring but this is about typesetting not maths so the distinction isn't important.}.
    There isn't a universally agreed upon standard notation for this, I have seen all of the following used:
    \begin{equation}
        \field_{m\times n}, \qquad \field^{m\times n}, \qquad \field^{m, n}, \qquad\text{and}\qquad \matrices[m]{n}{\field}.
    \end{equation}
    
    It is the last of these that I like the most as it fits with the notation of the other sets of matrices.
    Therefore this is the notation that I have implemented.
    To typeset \(\matrices[m]{n}{\field}\) use
    \begin{Verbatim}[gobble=2]
        \matrices[m]{n}{\field}
    \end{Verbatim}
    If we leave off the optional argument, as in \verb|\matrices{n}{\field}|, then we get \(\matrices{n}{\field}\), which is understood as shorthand for \(\matrices[n]{n}{\field}\), which are the square matrices.
    As well as these the following groups are defined: general linear group, special linear group, orthogonal group, special orthogonal group, unitary group, special unitary group.
    These are typeset using
    \begin{Verbatim}[gobble=2]
        \generalLinear    \specialLinear
        \orthogonal       \specialOrthogonal
        \unitary          \specialUnitary
    \end{Verbatim}
    and they will look like
    \begin{equation}
        \generalLinear, \qquad \specialLinear, \qquad \orthogonal, \qquad \specialOrthogonal, \qquad \unitary, \qquad\text{and}\qquad \specialUnitary.
    \end{equation}
    There are also Lie algebra's of these sets which are typeset with
    \begin{Verbatim}[gobble=2]
        \generalLinearLie    \specialLinearLie
        \orthogonalLie       \specialOrthogonalLie
        \unitaryLie          \specialUnitaryLie
    \end{Verbatim}
    which produces
    \begin{equation}
        \generalLinearLie, \qquad \specialLinearLie, \qquad \orthogonalLie, \qquad \specialOrthogonalLie, \qquad \unitaryLie, \qquad\text{and}\qquad \specialUnitaryLie.
    \end{equation}
    
    \section{Bra-Ket Notation}
    In quantum mechanics it is customary to denote vectors by \(\ket{\psi}\) and dual vectors by \(\bra{\varphi}\).
    These are called kets and bras respectively.
    We then write the inner product as \(\braket{\varphi}{\psi}\), which can be read as `bra-ket' or `bracket'\footnote{blame Paul Dirac}.
    In this package there are four related commands and four starred versions.
    The commands are
    \begin{Verbatim}[gobble=2]
        \ket{}        \bra{}        \braket{}{}        \ketbra{}{}
    \end{Verbatim}
    which, when given arguments, produce
    \begin{equation}
        \ket{\psi}, \qquad \bra{\varphi}, \qquad \braket{\varphi}{\psi}, \qquad\text{and}\qquad \ketbra{\psi}{\varphi}.
    \end{equation}
    The starred versions of these commands,
    \begin{Verbatim}[gobble=2]
        \ket*{}        \bra*{}        \braket*{}{}        \ketbra*{}{}
    \end{Verbatim}
    produce the same but scale with their arguments.
    For example
    \begin{equation}
        \ket*{\frac{1}{2}}, \qquad \bra*{\frac{1}{2}}, \qquad \braket*{\frac{1}{2}}{\frac{1}{2}}, \qquad\text{and}\qquad \ketbra*{\frac{1}{2}}{\frac{1}{2}}.
    \end{equation}
    This is \emph{not} the default behaviour because often we have arguments that are just slightly different in height and this produces multiple heights in the same equation.
    This is ugly.
    So the starred version is there only for when you explicitly want scaling.
    
    \section{Vectors}
    \subsection{Vectors Proper}
    There are many notations for vectors.
    Some of the most common are
    \begin{equation}
        \vec{a}, \qquad \underline{a}, \qquad \overline{a}, \qquad\text{and}\qquad \bm{a}.
    \end{equation}
    It is the last of these that I prefer and that is the default.
    The most basic vector command is \verb*|\vv{}| which is simply an alias for the \verb*|\bm| command from \package{bm} \cite{bm}.
    This can be redefined but it will typically cause issues.
    For example to redefine vectors to use an arrow above you would also have to think carefully about unit vectors with a hat.
    Speaking of unit vectors the following commands are all the vector commands defined:
    \begin{Verbatim}[gobble=2]
        \vv{x}        \vh{x}        \ve{x}
    \end{Verbatim}
    Which produce the following when given the argument \verb*|x|:
    \begin{equation}
        \vv{x}, \qquad \vh{x}, \qquad\text{and}\qquad \ve{x}.
    \end{equation}
    
    Note that the command \verb*|\vv| is redefined if it is already defined.
    For example \package{newtxmath} \cite{newtx} defines \verb*|\vv{AB}| to give \(\OLDvv{AB}\).
    This can still be accessed using \verb*|\OLDvv| in place of \verb*|\vv|.
    
    \subsection{Vector Calculus}
    To go along with vectors the four common vector calculus operators are defined, namely
    \begin{Verbatim}[gobble=2]
        \grad        \div        \curl        \laplacian
    \end{Verbatim}
    produce
    \begin{equation}
        \grad, \qquad \div, \qquad \curl, \qquad\text{and}\qquad \laplacian.
    \end{equation}
    Sometimes when working with multiple coordinate systems we use a subscript to denote which coordinate system the derivatives are with respect to.
    This can be achieved using the following
    \begin{Verbatim}[gobble=2]
        \grad[r]        \div[r]        \curl[r]        \laplacian[r]
    \end{Verbatim}
    which produce
    \begin{equation}
        \grad[r], \qquad \div[r], \qquad \curl[r], \qquad\text{and}\qquad \laplacian[r].
    \end{equation}
    Personally I prefer the basic undecorated \(\nabla\) for these but some people like to remind themselves that this is a vector\footnote{it isn't.}.
    For this use
    \begin{Verbatim}[gobble=2]
        \renewcommand{\nablaVector}{\vv{\nabla}}
        \renewcommand{\nablaVector}{\vec{\nabla}}
    \end{Verbatim}
    which will give \(\vv{\nabla}\) and \(\vec{\nabla}\) respectively.
    This will automatically apply to the gradient, divergence, and curl, but not the Laplacian as that is a scalar operator.
    To change the Laplacian instead redefine \verb*|\nablaScalar|.
    
    Sometimes one wishes to use the words grad, div, or curl, as operators, particularly when applied to objects other than standard \(\reals^n\) vectors.
    For this I have defined
    \begin{Verbatim}[gobble=2]
        \gradWord        \divWord        \curlWord
    \end{Verbatim}
    These produce
    \begin{equation}
        \gradWord, \qquad \divWord, \qquad\text{and}\qquad \curlWord
    \end{equation}
    These are typeset using \verb*|\@diffWordOperator|\footnote{if using @ in a command you either need to be in a \texttt{.sty} file or it needs to be wrapped in \texttt{\textbackslash{}makeatletter} and \texttt{\textbackslash{}makeatother}} and so you can change this to change how they are all typeset.
    Don't forget to include \verb*|\mathop| for proper spacing!
    I haven't done this for Laplacian as I've never seen it used.
    
    \section{Calculus}
    \subsection{Diffcoeff}
    The package \package{diffcoeff} \cite{diffcoeff} is loaded by \package{NotesMath}. 
    The default symbol is changed to an upright \enquote{d} from italic.
    The command \verb!\dd! is defined which produces output similar to \verb!\dl! but with a default spacing of 2mu in front of the differential. For example,
    \begin{Verbatim}[gobble=2]
        \diff[n]{f}{x}    \diff*{\diffp{\mathcal{L}}{\dot{q}}}{t}
        \dl{V} = \dl{x}\dd{y}\dd{z} = \dl{^3x_i}
    \end{Verbatim}
    produces
    \begin{equation}
        \diff[n]{f}{x},\qquad \diff*{\diffp{\mathcal{L}}{\dot{q}}}{t},\qquad \dl{V} = \dl{x}\dd{y}\dd{z} = \dl{^3x_i}
    \end{equation}
    As well as these new derivatives are declared:
    \begin{Verbatim}[gobble=2]
        \diffm{\vv{u}}{t}    \diffD{y}{x}
    \end{Verbatim}
    which produce
    \begin{equation}
        \diffm{\vv{u}}{t}, \qquad \diffD{y}{x}
    \end{equation}
    
    \section{Miscellaneous}
    The commands in this section don't fit in any other section.
    
    \subsection{Absolute Value}
    The absolute value command is \verb*|\abs{x}| which produces \(\abs{x}\).
    This will automatically resize with its argument.
    There is a starred version, \verb*|\abs*|, which doesn't resize.
    The following uses \verb*|\abs| and \verb*|\abs*| respectively:
    \begin{equation}
        \abs{\frac{1}{2}}, \qquad\text{and}\qquad \abs*{\frac{1}{2}}.
    \end{equation}
    Note that the resizing behaviour is opposite to bra-ket notation.
    This is because I do want the default behaviour of \verb*|\abs| to be resizing.
    
    \subsection{Fourier Transforms}
    The symbol for a Fourier transform can be typeset using \verb*|\fourierTransform| which gives \(\fourierTransform\).
    Similarly \verb*|\inverseFourierTransform| gives \(\inverseFourierTransform\).
    Both of the Fs in these are typeset using \verb*|\fourierTransform| so changing the forward transform will automatically change the inverse transform.
    
    \subsection{Wordy Operators}
    The diagonal operator, \(\diag\), is used to define a diagonal matrix by simply listing the elements on the diagonal.
    It can be typeset using \verb*|\diag|.
    
    The sign operator, \(\sgn\), returns the sign of its argument and can be typeset using \verb*|\sgn|.
    
    The trig function \(\cosec x = 1/\sin x\) is left out of \LaTeX which uses \(\csc\) instead.
    It can be typeset with \verb*|\cosec|.
    Similarly \(\cosech x = 1/\sinh x\) can be typeset with \verb*|\cosech|
    
    \subsection{Averages}
    Two commonly used notations for the mean/expected value of a random variable, \(x\), are \(\mean{x}\) and \(\expected{x}\).
    These can be typeset using \verb*|\mean{x}| and \verb*|\expected{x}| respectively.
    Additionally \verb*|\expected*| will resize with its argument.
    The following are typeset with \verb*|\expected| and \verb*|\expected*| respectively:
    \begin{equation}
        \expected{\frac{1}{2}}, \qquad\text{and}\qquad \expected*{\frac{1}{2}}.
    \end{equation}

    \subsection{Operators}
    Operators in quantum mechanics are often denoted with a hat.
    This can be typeset using \verb*|\operator{O}| which produces \(\operator{O}\).
    Sometimes operators can be seen as vectors, for example the angular momentum, in which case use \verb*|\vecoperator{L}| to produce \(\vecoperator{L}\).
    This is also aliased as \verb*|\operatorvec|.
    
    \chapter{NotesBoxes}
    The \package{NotesBoxes} package provides predefined boxes from the \package{tcolorbox} package \cite{tcolorbox}.
    
    \package{NotesBoxes} loads the following packages: \package{listings} \cite{listings}, \package{tcolorbox} \cite{tcolorbox}, \package{xcolor} \cite{xcolor} and \package{xparse} \cite{xparse}.
    As well, the following \package{tcolorbox} libraries are loaded: \package{skins}, \package{theorems}, \package{breakable}, and \package{hooks}.
    It is also assumed that \package{cleveref} is loaded \cite{cleveref}.
    
    Much of the code for these boxes was taken from \cite{Gleason}.
    
    \section{Important Box}
    The important box is the simplest box.
    It's use, as the name suggests, is to highlight important material.
    For example
    \begin{Verbatim}[gobble=2]
        \begin{important}
            The volume element in three dimensions is
            \begin{equation}
                \dd{V} = \dd{x}\dd{y}\dd{z} 
                = r^2\sin\vartheta \dd{r}\dd{\vartheta}\dd{\varphi}
            \end{equation}
        \end{important}
    \end{Verbatim}
    produces
    \begin{important}
        The volume element in three dimensions is
        \begin{equation}
            \dd{V} = \dd{x}\dd{y}\dd{z} 
            = r^2\sin\vartheta \dd{r}\dd{\vartheta}\dd{\varphi}.
        \end{equation}
    \end{important}
    
    \section{Theorem Box}
    The theorem box is for typesetting theorems and their proofs.
    For example
    \begin{Verbatim}[gobble=2]
        \begin{thm}{}{}
            There are infinitely many primes.
            \begin{proof}
                Suppose there is a finite number of primes, \(\{p_i\}\).
                Then \(\prod_i p_i + 1\) is either a new prime or has a
                prime factor not in \(\{p_i\}\).
                Therefore there are an infinite number of primes.
            \end{proof}
        \end{thm}
    \end{Verbatim}
    produces
    \begin{thm}{}{}
        There are infinitely many primes.
        \begin{proof}
            Suppose there is a finite number of primes, \(\{p_i\}\).
            Then \(\prod_i p_i + 1\) is either a new prime or has a prime factor not in \(\{p_i\}\).
            Therefore there are an infinite number of primes.
        \end{proof}
    \end{thm}
    Notice the two empty arguments.
    The first of these is for assigning a title to the theorem.
    The second is for assigning a label.
    This pattern follows for the rest of the boxes so won't be mentioned again.
    For example,
    \begin{Verbatim}[gobble=2]
        \begin{thm}{Title}{thm:label}
            A theorem.
        \end{thm}
    \end{Verbatim}
    produces
    \begin{thm}{Title}{thm:label}
        A theorem.
    \end{thm}
    which we can then reference using \verb*|\cref{thm:label}| or \verb*|\nameref{thm:label}| which produces \cref{thm:label} or \nameref{thm:label} respectively.
    
    \section{Almost Theorem Boxes}
    Often a result is not important enough to achieve the name of theorem.
    Instead such results are called claims, propositions, lemmas, or corollaries.
    There are boxes for these.
    \begin{Verbatim}[gobble=2]
        \begin{clm}{}{}
            This is true.
            \begin{proof}
                Just believe me.
            \end{proof}
        \end{clm}
    \end{Verbatim}
    produces
    \begin{clm}{}{}
        This is true.
        \begin{proof}
            Just believe me.
        \end{proof}
    \end{clm}
    and
    \begin{Verbatim}[gobble=2]
        \begin{prp}{}{}
            A thing that we think is true.
            \begin{proof}
                Trivial.
            \end{proof}
        \end{prp}
    \end{Verbatim}
    produces
    \begin{prp}{}{}
        A thing that we think is true.
        \begin{proof}
            Trivial.
        \end{proof}
    \end{prp}
    A lemma can be produced using
    \begin{Verbatim}[gobble=2]
        \begin{lma}{Jordan's Lemma}{lma:Jordan's lemma}
            Let \(f\) be a continuous complex function
            defined on the semicircle,
            \begin{equation}
                C_R = \{Re^{i\vartheta} \vert \vartheta \in [0, \pi]\},
            \end{equation}
            which has radius \(R > 0\) with the semicircle lying in the
            upper half plane and centred at the origin.
            If we can write \(f(z) = e^{iaz}g(z)\) for \(z\in C_R\) then
            \begin{equation}
                \abs{\int_{C_R} f(z) \dd{z}} \le \frac{\pi}{a}M_R
            \end{equation}
            where
            \begin{equation}
                M_R \coloneq \max_{\vartheta \in [0, \pi]}
                \abs{g(Re^{i\vartheta})}.
            \end{equation}
            \begin{proof}
                Left to the reader.
            \end{proof}
        \end{lma}
    \end{Verbatim}
    which produces
    \begin{lma}{Jordan's Lemma}{lma:Jordan's lemma}
        Let \(f\) be a continuous complex function defined on the semicircle,
        \begin{equation}
            C_R = \{Re^{i\vartheta} \vert \vartheta \in [0, \pi]\},
        \end{equation}
        which has radius \(R > 0\) with the semicircle lying in the
        upper half plane and centred at the origin.
        If we can write \(f(z) = e^{iaz}g(z)\) for \(z\in C_R\) then
        \begin{equation}
            \abs{\int_{C_R} f(z) \dd{z}} \le \frac{\pi}{a}M_R
        \end{equation}
        where
        \begin{equation}
            M_R \coloneq \max_{\vartheta \in [0, \pi]}\abs{g(Re^{i\vartheta})}.
        \end{equation}
        \begin{proof}
            Left to the reader.
        \end{proof}
    \end{lma}
    Similarly one may consider a corollary
    \begin{Verbatim}[gobble=2]
        \begin{crl}{}{}
            Let \(f(z) = 1/z\).
            Then
            \begin{equation}
                \int_{C_R} f(z) \dd{z} = 0
            \end{equation}
            where \(C_R\) is the semicircle of radius \(R\) in the upper half plane.
            Then
            \begin{equation}
                \lim_{R \to \infty} \int_{C_R} f(z) \dd{z} = 0.
            \end{equation}
            \begin{proof}
                The proof follows immediately from \nameref{lma:Jordan's lemma}.
            \end{proof}
        \end{crl}
    \end{Verbatim}
    produces
    \begin{crl}{}{}
        Let \(f(z) = 1/z\).
        Then
        \begin{equation}
            \int_{C_R} f(z) \dd{z} = 0
        \end{equation}
        where \(C_R\) is the semicircle of radius \(R\) in the upper half plane.
        Then
        \begin{equation}
            \lim_{R \to \infty} \int_{C_R} f(z) \dd{z} = 0.
        \end{equation}
        \begin{proof}
            The proof follows immediately from \nameref{lma:Jordan's lemma}.
        \end{proof}
    \end{crl}
    
    \section{Definitions}
    When defining something the box to use is the definition box.
    \begin{Verbatim}[gobble=2]
        \begin{dfn}{Group}{}
            A group, \(\langle G, \cdot \rangle\) is a set, \(G\), 
            and binary operation, \(\cdot\colon G\times G \to G\),
            which satisfies the following group axioms:
            \begin{align}
                \forall a, b, c\in G \vert a(bc) = (ab)c,\\
                \exists e \in G \vert eg = ge = e \forall g\in G,\\
                \forall g \in G \exists g^{-1} \in G \vert
                gg^{-1} = g^{-1}g = e.
            \end{align}
       \end{dfn}
    \end{Verbatim}
    produces
    \begin{dfn}{Group}{}
        A group, \(\langle G, \cdot \rangle\) is a set, \(G\), 
        and binary operation, \(\cdot\colon G\times G \to G\),
        which satisfies the following group axioms:
        \begin{align}
            \forall a, b, c\in G \vert a(bc) = (ab)c,\\
            \exists e \in G \vert eg = ge = e \forall g\in G,\\
            \forall g \in G \exists g^{-1} \in G \vert 
            gg^{-1} = g^{-1}g = e.
        \end{align}
    \end{dfn}

    We may wish to introduce some new notation.
    This can be done with the notation box:
    \begin{Verbatim}[gobble=2]
        \begin{ntn}{}{}
            We write \(A \cong B \iff \) \(A\) is isomorphic to \(B\).
        \end{ntn}
    \end{Verbatim}
    produces
    \begin{ntn}{}{}
        We write \(A \cong B \iff \) \(A\) is isomorphic to \(B\).
    \end{ntn}

    \section{Additional Material Boxes}
    Suppose we wish to give an example.
    Then we can use the example box:
    \begin{Verbatim}[gobble=2]
        \begin{exm}{}{}
            The set \(\integers/2\integers = \{0, 1\}\)
            forms a group under addition modulo 2.
        \end{exm}
    \end{Verbatim}
    \begin{exm}{}{}
        The set \(\integers/2\integers = \{0, 1\}\) forms a group
        under addition modulo 2.
    \end{exm}
    Or suppose that we want to note an application of what we have derived.
    Then we can use the application box.
    \begin{Verbatim}[gobble=2]
        \begin{app}{}{}
            Groups are useful in physics as they describe the 
            symmetries of an object
        \end{app}
    \end{Verbatim}
    \begin{app}{}{}
        Groups are useful in physics as they describe the 
        symmetries of an object
    \end{app}
    
    \section{Code Boxes}
    If we want to display code then we can do this in a code block.
    \begin{Verbatim}[gobble=2]
        \begin{cde}{}{}
            \begin{lstlisting}[language=python, gobble=12]
                def foo(x):
                    print(f"x = {x}")
            \end{lstlisting}
        \end{cde}
    \end{Verbatim}
    produces
    \begin{cde}{}{}
        \begin{lstlisting}[language=python, gobble=12]
            def foo(x):
                print(f"x = {x}")
        \end{lstlisting}
    \end{cde}
    The code is typeset using the \verb*|lstlistings| environment from the \package{listings} package so see \cite{listings} for more details on that.
    Note that the \verb*|gobble=n| keyword removes the first \verb*|n| characters.
    In my document I have 12 spaces of indentation within the \verb*|lstlisting| environment and so they are removed by \verb*|gobble|.
    
    The colour scheme is based on the solarized colour scheme \cite{solarized}.
    It should work for multiple languages.
    For example
    \begin{Verbatim}[gobble=2]
        \begin{cde}{}{}
            \begin{lstlisting}[language=C++, gobble=12]
                #include <iostream>
                template <typename T>
                void foo(T x) {
                    std::cout << "x = " << x << std::endl;
                }
            \end{lstlisting}
        \end{cde}
    \end{Verbatim}
    produces
    \begin{cde}{}{}
        \begin{lstlisting}[language=C++, gobble=12]
            #include <iostream>
            template <typename T>
            void foo(T x) {
                std::cout << "x = " << x << std::endl;
            }
        \end{lstlisting}
    \end{cde}
    and
    \begin{Verbatim}[gobble=2]
        \begin{cde}{}{}
            \begin{lstlisting}[language={[LaTeX]TeX}, gobble=12]
                \newcommand{\foo}[1]{#1}
            \end{lstlisting}
        \end{cde}
    \end{Verbatim}
    produces
    \begin{cde}{}{}
        \begin{lstlisting}[language={[LaTeX]TeX}, gobble=12]
            \newcommand{\foo}[1]{#1}
        \end{lstlisting}
    \end{cde}

    \section{Remark and Warn Boxes}
    Often we wish to make a note of something without breaking the flow or maybe draw attention to an important point.
    For this we have the remark and warn boxes.
    For example
    \begin{Verbatim}[gobble=2]
        \begin{rmk}
            Notice that ...
        \end{rmk}
    \end{Verbatim}
    produces
    \begin{rmk}
        Notice that ...
    \end{rmk}
    Similarly
    \begin{Verbatim}[gobble=2]
        \begin{wrn}
            Be careful not to ...
        \end{wrn}
    \end{Verbatim}
    produces
    \begin{wrn}
        Be careful not to ...
    \end{wrn}
    
    We can put boxes in boxes.
    This is most useful when putting remarks, warnings, or important boxes in other boxes.
    For example
    \begin{exm}{}{}
        This is an example
        \begin{rmk}
            This is a remark    
        \end{rmk}
    \end{exm}
    \begin{lma}{}{}
        This is a lemma
        \begin{wrn}
            This is a warning
        \end{wrn}
    \end{lma}
    \begin{thm}{}{}
        This is a theorem
        \begin{important}
            This is important
        \end{important}
    \end{thm}
    In theory we can go multiple levels deep but this starts to look silly.
    
    \begin{dfn}{}{}
        If we have long boxes they should automatically break over the page.
        For example this definition has a list of properties:
        \begin{enumerate}
            \item First property
            \item Second property
            \item Third property
        \end{enumerate}
        and lots of text which means it takes more space than is left on the page.
        
        Lets test this by making the text in this box so long that it has no choice but to break into two parts.
        It's becoming hard to think of something to say so I'm just going to use lorem ipsum.
        
        \blindtext
        
        Hey Look!
        We're on the next page.
    \end{dfn}{}{}
    \backmatter
    \printbibliography
\end{document}
